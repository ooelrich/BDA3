\documentclass[]{article}
\usepackage{lmodern}
\usepackage{amssymb,amsmath}
\usepackage{ifxetex,ifluatex}
\usepackage{fixltx2e} % provides \textsubscript
\ifnum 0\ifxetex 1\fi\ifluatex 1\fi=0 % if pdftex
  \usepackage[T1]{fontenc}
  \usepackage[utf8]{inputenc}
\else % if luatex or xelatex
  \ifxetex
    \usepackage{mathspec}
  \else
    \usepackage{fontspec}
  \fi
  \defaultfontfeatures{Ligatures=TeX,Scale=MatchLowercase}
\fi
% use upquote if available, for straight quotes in verbatim environments
\IfFileExists{upquote.sty}{\usepackage{upquote}}{}
% use microtype if available
\IfFileExists{microtype.sty}{%
\usepackage{microtype}
\UseMicrotypeSet[protrusion]{basicmath} % disable protrusion for tt fonts
}{}
\usepackage[margin=1in]{geometry}
\usepackage{hyperref}
\hypersetup{unicode=true,
            pdftitle={Exercise 1.7},
            pdfauthor={Oscar Oelrich},
            pdfborder={0 0 0},
            breaklinks=true}
\urlstyle{same}  % don't use monospace font for urls
\usepackage{graphicx,grffile}
\makeatletter
\def\maxwidth{\ifdim\Gin@nat@width>\linewidth\linewidth\else\Gin@nat@width\fi}
\def\maxheight{\ifdim\Gin@nat@height>\textheight\textheight\else\Gin@nat@height\fi}
\makeatother
% Scale images if necessary, so that they will not overflow the page
% margins by default, and it is still possible to overwrite the defaults
% using explicit options in \includegraphics[width, height, ...]{}
\setkeys{Gin}{width=\maxwidth,height=\maxheight,keepaspectratio}
\IfFileExists{parskip.sty}{%
\usepackage{parskip}
}{% else
\setlength{\parindent}{0pt}
\setlength{\parskip}{6pt plus 2pt minus 1pt}
}
\setlength{\emergencystretch}{3em}  % prevent overfull lines
\providecommand{\tightlist}{%
  \setlength{\itemsep}{0pt}\setlength{\parskip}{0pt}}
\setcounter{secnumdepth}{0}
% Redefines (sub)paragraphs to behave more like sections
\ifx\paragraph\undefined\else
\let\oldparagraph\paragraph
\renewcommand{\paragraph}[1]{\oldparagraph{#1}\mbox{}}
\fi
\ifx\subparagraph\undefined\else
\let\oldsubparagraph\subparagraph
\renewcommand{\subparagraph}[1]{\oldsubparagraph{#1}\mbox{}}
\fi

%%% Use protect on footnotes to avoid problems with footnotes in titles
\let\rmarkdownfootnote\footnote%
\def\footnote{\protect\rmarkdownfootnote}

%%% Change title format to be more compact
\usepackage{titling}

% Create subtitle command for use in maketitle
\newcommand{\subtitle}[1]{
  \posttitle{
    \begin{center}\large#1\end{center}
    }
}

\setlength{\droptitle}{-2em}

  \title{Exercise 1.7}
    \pretitle{\vspace{\droptitle}\centering\huge}
  \posttitle{\par}
    \author{Oscar Oelrich}
    \preauthor{\centering\large\emph}
  \postauthor{\par}
      \predate{\centering\large\emph}
  \postdate{\par}
    \date{12 december 2018}


\begin{document}
\maketitle

In this exercise we will use Bayes theorem to solve the Monty Hall
problem. The key to solving it is in how we chose to condition. Without
loss of generality we call the door we choose \textbf{A}, the door Monty
opens \textbf{B} and the last door (that we can potentially switch to)
door \textbf{C}. We want to calculate the probability of winning when
using the two different strategies available to us (switching and not
switching). Since we picked door \textbf{A}, the probability we are
interested in when we want to find the probability of winning using the
``not switch'' strategy is the probability that the car is behind door
\textbf{A} given that Monty opened door \textbf{B}. We use \(M(-)\) to
denote Monty opening door -, and \(C(-)\) for the car being behind door
-.

\[\begin{equation}
P(C(A)|M(B))=\frac{P(M(B)|C(A))P(C(A))}{P(M(B))}
\end{equation}\]

To calculate this probability we need to calculate the following.

\begin{itemize}
\tightlist
\item
  \(P(M(B)|C(A))\), which is the probability that Monty opens door
  \textbf{B} given that the car is behind door \textbf{A}. Since the car
  is behind \textbf{A}, Monty can chose between door \textbf{B} and
  \textbf{C} freely, and so this probability is \(0.5\).
\item
  The unconditional probability that the car is behind any particular
  door is \(P(C(C))=\frac{1}{3}\)
\item
  The unconditional probability that Monty opens door \textbf{B} can be
  obtained by summing over the conditional probabilities for the car
  being behind the different doors. If the car is being door \textbf{B}
  it is zero, if the car is behind door \textbf{C} it is one and if the
  car is behind door \textbf{A} it is one half, with the probability of
  the car being behind any particular door being one third for each
  door, this sums up to one \(\frac{1}{2}\).
\end{itemize}

Plugging in the values into Bayes theorem we obtain

\[\begin{equation}
P(C(A)|M(B))=\frac{\frac{1}{2}\frac{1}{3}}{\frac{1}{2}}=\frac{1}{3}
\end{equation}\]

If we instead use the ``switching'' strategy, we only need to calculate
one new probability, namely \(P(M(B)|C(C))\), since now we are
interested in \(P(C(C)|M(B))\) (which is the probability of winning by
using the switching strategy, we win if the car is behind the door we
did not chose and that Monty did not open). Given that the car is behind
door \textbf{C}, Monty has no choice but to open door \textbf{B} so this
probability is one. This gives us a chance of winning equal to

\[\begin{equation}
P(C(C)|M(B))=\frac{1\frac{1}{3}}{\frac{1}{2}}=\frac{2}{3}
\end{equation}\]

Note that we could have obtained this by symmetry. Since given that
Monty opens door \textbf{B}, the car is either behind door \textbf{A} or
\textbf{C}. So if we calculate the probability of it being behind
\textbf{A} to be one third, then the probability that it is behind door
\textbf{C} is clearly two thirds.


\end{document}
