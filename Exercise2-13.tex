\documentclass[]{article}
\usepackage{lmodern}
\usepackage{amssymb,amsmath}
\usepackage{ifxetex,ifluatex}
\usepackage{fixltx2e} % provides \textsubscript
\ifnum 0\ifxetex 1\fi\ifluatex 1\fi=0 % if pdftex
  \usepackage[T1]{fontenc}
  \usepackage[utf8]{inputenc}
\else % if luatex or xelatex
  \ifxetex
    \usepackage{mathspec}
  \else
    \usepackage{fontspec}
  \fi
  \defaultfontfeatures{Ligatures=TeX,Scale=MatchLowercase}
\fi
% use upquote if available, for straight quotes in verbatim environments
\IfFileExists{upquote.sty}{\usepackage{upquote}}{}
% use microtype if available
\IfFileExists{microtype.sty}{%
\usepackage{microtype}
\UseMicrotypeSet[protrusion]{basicmath} % disable protrusion for tt fonts
}{}
\usepackage[margin=1in]{geometry}
\usepackage{hyperref}
\hypersetup{unicode=true,
            pdftitle={Exercise 2.13},
            pdfauthor={Oscar Oelrich},
            pdfborder={0 0 0},
            breaklinks=true}
\urlstyle{same}  % don't use monospace font for urls
\usepackage{color}
\usepackage{fancyvrb}
\newcommand{\VerbBar}{|}
\newcommand{\VERB}{\Verb[commandchars=\\\{\}]}
\DefineVerbatimEnvironment{Highlighting}{Verbatim}{commandchars=\\\{\}}
% Add ',fontsize=\small' for more characters per line
\usepackage{framed}
\definecolor{shadecolor}{RGB}{248,248,248}
\newenvironment{Shaded}{\begin{snugshade}}{\end{snugshade}}
\newcommand{\AlertTok}[1]{\textcolor[rgb]{0.94,0.16,0.16}{#1}}
\newcommand{\AnnotationTok}[1]{\textcolor[rgb]{0.56,0.35,0.01}{\textbf{\textit{#1}}}}
\newcommand{\AttributeTok}[1]{\textcolor[rgb]{0.77,0.63,0.00}{#1}}
\newcommand{\BaseNTok}[1]{\textcolor[rgb]{0.00,0.00,0.81}{#1}}
\newcommand{\BuiltInTok}[1]{#1}
\newcommand{\CharTok}[1]{\textcolor[rgb]{0.31,0.60,0.02}{#1}}
\newcommand{\CommentTok}[1]{\textcolor[rgb]{0.56,0.35,0.01}{\textit{#1}}}
\newcommand{\CommentVarTok}[1]{\textcolor[rgb]{0.56,0.35,0.01}{\textbf{\textit{#1}}}}
\newcommand{\ConstantTok}[1]{\textcolor[rgb]{0.00,0.00,0.00}{#1}}
\newcommand{\ControlFlowTok}[1]{\textcolor[rgb]{0.13,0.29,0.53}{\textbf{#1}}}
\newcommand{\DataTypeTok}[1]{\textcolor[rgb]{0.13,0.29,0.53}{#1}}
\newcommand{\DecValTok}[1]{\textcolor[rgb]{0.00,0.00,0.81}{#1}}
\newcommand{\DocumentationTok}[1]{\textcolor[rgb]{0.56,0.35,0.01}{\textbf{\textit{#1}}}}
\newcommand{\ErrorTok}[1]{\textcolor[rgb]{0.64,0.00,0.00}{\textbf{#1}}}
\newcommand{\ExtensionTok}[1]{#1}
\newcommand{\FloatTok}[1]{\textcolor[rgb]{0.00,0.00,0.81}{#1}}
\newcommand{\FunctionTok}[1]{\textcolor[rgb]{0.00,0.00,0.00}{#1}}
\newcommand{\ImportTok}[1]{#1}
\newcommand{\InformationTok}[1]{\textcolor[rgb]{0.56,0.35,0.01}{\textbf{\textit{#1}}}}
\newcommand{\KeywordTok}[1]{\textcolor[rgb]{0.13,0.29,0.53}{\textbf{#1}}}
\newcommand{\NormalTok}[1]{#1}
\newcommand{\OperatorTok}[1]{\textcolor[rgb]{0.81,0.36,0.00}{\textbf{#1}}}
\newcommand{\OtherTok}[1]{\textcolor[rgb]{0.56,0.35,0.01}{#1}}
\newcommand{\PreprocessorTok}[1]{\textcolor[rgb]{0.56,0.35,0.01}{\textit{#1}}}
\newcommand{\RegionMarkerTok}[1]{#1}
\newcommand{\SpecialCharTok}[1]{\textcolor[rgb]{0.00,0.00,0.00}{#1}}
\newcommand{\SpecialStringTok}[1]{\textcolor[rgb]{0.31,0.60,0.02}{#1}}
\newcommand{\StringTok}[1]{\textcolor[rgb]{0.31,0.60,0.02}{#1}}
\newcommand{\VariableTok}[1]{\textcolor[rgb]{0.00,0.00,0.00}{#1}}
\newcommand{\VerbatimStringTok}[1]{\textcolor[rgb]{0.31,0.60,0.02}{#1}}
\newcommand{\WarningTok}[1]{\textcolor[rgb]{0.56,0.35,0.01}{\textbf{\textit{#1}}}}
\usepackage{longtable,booktabs}
\usepackage{graphicx,grffile}
\makeatletter
\def\maxwidth{\ifdim\Gin@nat@width>\linewidth\linewidth\else\Gin@nat@width\fi}
\def\maxheight{\ifdim\Gin@nat@height>\textheight\textheight\else\Gin@nat@height\fi}
\makeatother
% Scale images if necessary, so that they will not overflow the page
% margins by default, and it is still possible to overwrite the defaults
% using explicit options in \includegraphics[width, height, ...]{}
\setkeys{Gin}{width=\maxwidth,height=\maxheight,keepaspectratio}
\IfFileExists{parskip.sty}{%
\usepackage{parskip}
}{% else
\setlength{\parindent}{0pt}
\setlength{\parskip}{6pt plus 2pt minus 1pt}
}
\setlength{\emergencystretch}{3em}  % prevent overfull lines
\providecommand{\tightlist}{%
  \setlength{\itemsep}{0pt}\setlength{\parskip}{0pt}}
\setcounter{secnumdepth}{0}
% Redefines (sub)paragraphs to behave more like sections
\ifx\paragraph\undefined\else
\let\oldparagraph\paragraph
\renewcommand{\paragraph}[1]{\oldparagraph{#1}\mbox{}}
\fi
\ifx\subparagraph\undefined\else
\let\oldsubparagraph\subparagraph
\renewcommand{\subparagraph}[1]{\oldsubparagraph{#1}\mbox{}}
\fi

%%% Use protect on footnotes to avoid problems with footnotes in titles
\let\rmarkdownfootnote\footnote%
\def\footnote{\protect\rmarkdownfootnote}

%%% Change title format to be more compact
\usepackage{titling}

% Create subtitle command for use in maketitle
\providecommand{\subtitle}[1]{
  \posttitle{
    \begin{center}\large#1\end{center}
    }
}

\setlength{\droptitle}{-2em}

  \title{Exercise 2.13}
    \pretitle{\vspace{\droptitle}\centering\huge}
  \posttitle{\par}
    \author{Oscar Oelrich}
    \preauthor{\centering\large\emph}
  \postauthor{\par}
      \predate{\centering\large\emph}
  \postdate{\par}
    \date{17 december 2018}


\begin{document}
\maketitle

\begin{longtable}[]{@{}llll@{}}
\toprule
Year & Fatal accidents & Passenger deaths & Death rate\tabularnewline
\midrule
\endhead
1976 & 24 & 734 & 0.19\tabularnewline
1977 & 25 & 516 & 0.12\tabularnewline
1978 & 31 & 754 & 0.15\tabularnewline
1979 & 31 & 877 & 0.16\tabularnewline
1980 & 22 & 814 & 0.14\tabularnewline
1981 & 21 & 362 & 0.06\tabularnewline
1982 & 26 & 764 & 0.13\tabularnewline
1983 & 20 & 809 & 0.13\tabularnewline
1984 & 16 & 223 & 0.03\tabularnewline
1985 & 22 & 1066 & 0.15\tabularnewline
\bottomrule
\end{longtable}

\hypertarget{part-a}{%
\subsubsection{Part a}\label{part-a}}

We assume that the number of fatal accidents in a year follows a Poisson
distribution with parameter \(\theta\), with probability mass function
given by

\[\begin{equation}
p(y|\theta) = \frac{\theta^y e^{-\theta}}{y!}.
\end{equation}\]

The conjugate prior of the Poisson is the gamma distribution, which is a
distribution with two parameters (\(\alpha\) and \(\beta\)), and density

\[\begin{equation}
p(\theta)=\frac{\beta^{\alpha}}{\Gamma(\alpha)} \theta^{\alpha-1}e^{-\beta \theta}.
\end{equation}\]

For ease of notation, we use \(\boldsymbol{y}\) to refer to the observed
data, and \(y_1986\) to refer to the (future) observation we wish to
predict. The likelihood for the ten observations is given by

\[\begin{equation}
L(\boldsymbol{y}|\theta)=\prod_{i=1}^{10}\frac{\theta^{y_i} e^{-\theta}}{y_i!} \propto \theta^{\sum_{i=1}^{10} y_i}e^{-n\theta}.
\end{equation}\]

Since we have a conjugate prior, we know that the posterior will be from
the same family (gamma), and since the gamma distribution is fully
identified by looking at the exponential term and the exponent of
\(\theta\) we can find it easily by multiplying the relevant parts of
the prior and likelihood

\[\begin{equation}
p(\theta|\boldsymbol{y}) \propto \theta^{\alpha-1}e^{-\beta \theta}\theta^{\sum_{i=1}^{10} y_i}e^{-n\theta}=\theta^{(\alpha + \sum_{i=1}^{10} y_i) - 1}e^{-(n+\beta)\theta},
\end{equation}\]

which tells us that the posterior must be a
\(gamma(\alpha + \sum_{i=1}^{10} y_i, n+\beta)\) distribution. Since we
really don't have any prior information we pick \(\alpha\) and \(\beta\)
to be arbitrarily small, and so our posterior is
\(gamma(\sum_{i=1}^{10} y_i, n)=gamma(238, 10)\).

Next we wish to use our posterior distribution to make a \(95 \%\)
predictive interval for the number of accidents the next year. The
posterior predictive distribution for the number of accidents next year
is given by

\[\begin{equation}
p(y_{1986}|\boldsymbol{y}) = \int_{\theta} p(y_{1986}|\boldsymbol{y}, (\theta|\boldsymbol{y})) p(\theta|\boldsymbol{y}) d\theta \\ \qquad \quad =\int_{\theta} p(y_{1986}|\theta) p(\theta|\boldsymbol{y}) d\theta. 
\end{equation}\]

We will calculate/approximate it using two different methods: by
simulation and by normal approximation.

The normal approximation approach uses the fact that if both the
posterior and the data (conditional on the posterior) follow normal
distributions then \(p(y_{1986}|\theta) p(\theta)=p(y_{1986},\theta)\)
is jointly normal and so the marginal distribution of the predicted
value is also normal. All we need to do is to find the mean and variance
of the distribution.

Using the law of total expectation, we find the mean to be

\[\begin{equation}
\mathbb{E}(y_{1986}|y_{1976:1985})=\mathbb{E}(\mathbb{E}(y_{1986}|y_{1976:1985},(\theta|y_{1976:1985}))=\mathbb{E}(\mathbb{E}(y_{1986}|(\theta|y_{1976:1985})),
\end{equation}\]

where the last step follows from the fact that the old observations only
influence the expected number of accidents in the future through the
posterior distribution. Further

\[\begin{equation}
\mathbb{E}(\mathbb{E}(y_{1986}|(\theta|y_{1976:1985}))=\mathbb{E}(\theta|y_{1976:1985})=\frac{238}{10}=23.8,
\end{equation}\]

where we use the fact that the expectation of the predictive
distribution is \(\theta\) (using the normal approximation of a
\(Poisson(\theta)\) distribution, we get that the data follows a
\(N(\theta, \theta)\) distribution, and thus has mean/expectation equal
to \(\theta\)), and that the mean of a \(gamma(\alpha, \beta)\) is
\(\frac{\alpha}{\beta}\).

In order to find the variance we need to use the formula for conditional
variance

\[\begin{align}
\mathrm{var}(y_{1986}|y_{1976:1985})&=\mathbb{E}_{\theta|y_{1976:1985}}(\mathrm{var}(y_{1986}|y_{1976:1985},(\theta|y_{1976:1985})))+\mathrm{var}_{\theta|y_{1976:1985}}(\mathbb{E}(y_{1986}|y_{1976:1985},(\theta|y_{1976:1985}))) \\
&= \mathbb{E}_{\theta|y_{1976:1985}}(\mathrm{var}(y_{1986}|y_{1976:1985}))+\mathrm{var}_{\theta|y_{1976:1985}}(\mathbb{E}(y_{1986}|(\theta|y_{1976:1985})))   \\
&=\mathbb{E}_{\theta|y_{1976:1985}}(\theta|y_{1976:1985})+\mathrm{var}_{\theta|y_{1976:1985}}(\theta|y_{1976:1985}) \\
&= 23.8 + 2.38 = 5.12^2, 
\end{align}\]

where again we use that the variance of the data is \(\theta\) and that
the variance of the gamma distribution is \(\frac{\alpha}{\beta^2}\). So
the posterior predictive distribution, using normal approximation, is
\(N(23.8, 5.12^2)\). From this we can easily construct a \(95 \%\)
predictive interval as \([23.8 \pm 5.12*1.96]=[13.8, 33.8]\) which,
considering that the number of accidents is a discrete number, gives the
interval \([13, 34]\).

The other way of solving the exercise is to use Monte Carlo simulation:
sample from the posterior distribution and then for each simulated value
generate an observation from a Poisson distribution with intensity equal
to the simulated value. We can then use the required percentiles of this
simulated data to find the \(95 \%\) predictive interval.

\begin{Shaded}
\begin{Highlighting}[]
\CommentTok{#  generate draws from the posterior}
\NormalTok{n_obs <-}\StringTok{ }\DecValTok{10000} \CommentTok{# number of simulations}
\NormalTok{theta <-}\StringTok{ }\KeywordTok{rgamma}\NormalTok{(n_obs, }\DecValTok{238}\NormalTok{, }\DecValTok{10}\NormalTok{)}
\NormalTok{y_gen <-}\StringTok{ }\KeywordTok{rpois}\NormalTok{(n_obs, theta)}
\KeywordTok{print}\NormalTok{(}\KeywordTok{sort}\NormalTok{(y_gen)[}\KeywordTok{c}\NormalTok{(}\FloatTok{0.025}\NormalTok{, }\FloatTok{0.975}\NormalTok{)}\OperatorTok{*}\NormalTok{n_obs]) }\CommentTok{# computed interval is (13, 34)}
\end{Highlighting}
\end{Shaded}

\begin{verbatim}
## [1] 14 34
\end{verbatim}

Quite close to our normal approximation.

\hypertarget{part-b}{%
\subsubsection{Part b}\label{part-b}}

\hypertarget{part-c}{%
\subsubsection{Part c}\label{part-c}}

\hypertarget{part-d}{%
\subsubsection{Part d}\label{part-d}}

\hypertarget{part-e}{%
\subsubsection{Part e}\label{part-e}}


\end{document}
